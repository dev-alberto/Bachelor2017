%
% File: chap01.tex
% Author: Victor F. Brena-Medina
% Description: Introduction chapter where the biology goes.
%
\let\textcircled=\pgftextcircled

\chapter{Structuri Algebrice de baza} 
\section{Grupuri}
\label{sec:sec01}
In aceasta sectiune vom pune bazele matematice pentru urmatoarele capitole, facand astfel o introducere a unor concepte fundamentale din Teoria Grupurilor , a Corpurilor si in special despre Corpuri finite, care au o importanta deosebita pentru subiectul acestei licente.
\begin{dfn}
Un \textit{grup} reprezinta o multime $S$ impreuna cu o operatie de compozitie $\cdot$ astfel incat sunt respectate urmatoarele reguli:
\\ - legea $\cdot$ e asociativa $\rightarrow$ $\forall x, y, z\in S: (x\cdot y)\cdot z = x\cdot (y \cdot z)$
\\ - existanta unui element neutru, unic $\rightarrow$ $\exists e\in S: x \cdot e = x, \forall x\in S$
\\ - $\forall x\in S, \exists y\in S$, numit invers al elementului $x$, astfel incat $x\cdot y = y\cdot x = e$
\end{dfn}

\begin{dfn}
Se numeste grup \textit{abelian}, acel grup in care legea de compozitie $\cdot$ este de asemenea comutativa, adica pentru orice doua elemente
$x, y \in S$ avem $x \cdot y = y \cdot x$
\end{dfn}

\begin{dfn}
Fie $G$ un grup si H o submultime a lui $G$. $H$ este un \textit{subgrup} a lui $G$ daca sunt indeplinite conditiile:
\\ -inchidere la operatia de compozitie $\rightarrow$ $\forall x, y \in H, x \cdot y \in H$
\\ - daca $x \in H \Rightarrow x^{-1} \in H$
\end{dfn}

\begin{dfn}
Se numeste \textit{ordin} al grupului $G$, cardinalul multimii G, notat $|G|$. Pentru un element $g \in G$, se numeste ordin al lui $g$, notat prin $ord_G(g)$, cel mai
mic intreg $n$ astfel incat $g^n = e$, unde $e$ este element netru al lui $G$. Altfel spunem ca ordinul lui $g$ este infinit.
\end{dfn}

\begin{dfn}
Fie $G$ un grup si $g \in G$. Se numeste subgrup generat de $g$ multimea $\set{g^n\mid n\in \mathbb{Z}}$ si se noteaza $\langle g \rangle$. $G = \langle g \rangle$ daca si numai daca $ord_G(g) = |G|$. Atunci $G$ se numeste grup ciclic iar $g$ generator pentru $G$.
\end{dfn}



%=======
\section{Inele}
\label{sec:sec02}

\begin{dfn}
Un \textit{inel} reprezinta un triplet $(R, +, \cdot)$, unde $R$ este o multime iar $+, \cdot$ reprezinta 2 legi de compozitie. Urmatoarele proprietati trebuiesc simultan indeplinite: 
\\ - $(R, +)$ reprezinta un grup abelian
\\ - $x \cdot y = y \cdot x, \forall x,y\in R$ si de asemenea exista element neutru la inmultire, diferit de elementul neutru la adunare
\\ - legea $\cdot$ este distributiva fata de $+$, adica $\forall x,y,z\in R, x\cdot (y + z) = x\cdot y + x\cdot z$ si $(y + z)\cdot x = y\cdot x + z\cdot x$ 
\end{dfn}

\begin{dfn}
Fie $R, R'$ doua inele cu operatiile $+, \times$ respectiv $\oplus, \otimes$. Un \textit{homomorfism} de inele este o functie $\Psi : R\rightarrow R'$ care este definita pentru orice doua elemente $x, y\in R$ astfel:
\\$\Psi (x+y) = \Psi (x) \oplus \Psi (y)$
\\$\Psi (x\times y) = \Psi (x) \otimes \Psi (y)$
\\$\Psi (1) = 1$
\end{dfn}

\begin{dfn}
Fie $R$ un inel. $I$ Se numeste \textit{ideal} (la stanga sau la dreapta) a lui $R$ daca $I\subseteq R, I \neq \emptyset$ si respecta:
\\ - $I$ este un subgrup pentru $(R, +)$
\\ - $\forall x\in R, \forall y\in I \Rightarrow x\cdot y, y\cdot x \in I$
\\ $I$ se numeste ideal bilateral daca este ideal la stanga si la dreapta.
\\ Idealul $I\subsetneq R$ este prim daca $\forall x, y\in R$ cu $x\cdot y\in I$ avem $x\in I \lor y \in I$
\\Idealul $J\subsetneq R$ este maximal daca pentru oricare alt ideal $J$ avem $J=I \lor J=R$
\end{dfn}

\begin{obs}
Fie $\Psi$ un homomorfism de la $\mathbb{Z}$ la un inel $R$ definit astfel:
\\ $\Psi (n) = \begin{cases} 
     1+...+1 & $de n ori daca $n\geq 0 \\
   -(1+...+1) & $de -n ori altfel $
   \end{cases}$
\end{obs}
 \textit{Nucleul} lui $\Psi$ este un ideal a lui $\mathbb{Z}$ si daca toti multipli de 1 sunt diferiti atunci $ker\Psi = {0}$. Altfel, daca $R$ este finit de exemplu, cativa multiplii vor fi $0$. Altfel spus, nucleul lui $\Psi$ este generat de un numar natural $m$
 
 \begin{dfn}
 Fie $R$ un inel si $\Psi$ definit ca mai sus. Nucleul lui $\Psi$ are forma $m\mathbb{Z}, m\in\mathbb{N}$. $m$ poarta denumirea de \textit{caracteristica} a inelului $R$ si se noteaza $char(R)$
 \end{dfn}
 
 \begin{dfn}
 Fie $R$ un inel. Un element $x\in R$ este inversabil daca $\exists! y\in R, x\cdot y = y\cdot x = e$. $y$ se numeste unitate. Multimea tuturor unitatilor se noteaza cu $R^{*}$
 \end{dfn}
 
 \begin{dfn}
 Fie $n\geq 1$. Notam cu $\phi (n) = |(\mathbb{Z}/n\mathbb{Z})^{*}|$. Functia $\phi$ se numeste \textit{functia lui Euler}. Avem $\phi (n) = |\set{x\mid 1\leq x\leq n, gcd(x, n)=1}|$
 \end{dfn}
 
 \begin{teo}
 Fie $n, x$ doua numere intregi astfel incat $gcd(n,x)=1$. Atunci este adevarata relatia:
  $x^{\phi(n)}\equiv 1 (mod n)$
 \end{teo}
 
 \section{Corpuri}
\label{sec:sec03}
\begin{dfn}
Un \textit{corp} $K$ este un inel comutativ cu toate elementele diferite de $0$ inversabile.
\end{dfn}

\begin{ex}
Multimea numerelor rationale $\mathbb{Q}$ impreuna cu operatiile obisnuite de adunare si inmultire este un corp. Pentru orice numar prim $p$, $\mathbb{Z}/p\mathbb{Z}$ este corp.  
\end{ex}

\begin{prop}
Caracteristica unui corp este $0$ sau $p$, un numar prim
\end{prop}

\begin{prop}
Fie $R$ un inel si I un ideal. Inelul factor $R/I$ este corp daca si numai daca $I$ este maximal.
\end{prop}

\begin{dfn}
Fie $K, K'$ doua corpuri. Un homomorfism de corpuri este un homomorfism de inele intre $K$ si $K'$. Remarcam faptul ca o astfel de functie este tot timpul injectiva deoarece nucleul acesteia este $\set {0}$ 
\end{dfn}

\begin{dfn}
Fie $L$ un corp. Un \textit{subcorp} a lui $L$ este o submultime $K$ a lui $L$ care isi pastreaza proprietatea de corp, operatiile aditive respectiv multiplicative fiind mostenite de la $L$. In aceasta situatie, corpul mare, $L$, se numeste o \textit{extensie} a corpului $K$.
\end{dfn}
\subsection{Corpuri Finite}
\label{subsec:subsec01}

In sistemele criptografice bazate pe curbe eliptice este important sa avem implementat operatii eficiente pe corpuri finite. Trei tipuri de corpuri finite reprezinta canditati potriviti pentru implementarea acestor operatii, respectiv corpurile prime, corpurile binare si corpurile de extensie optimale. In randurile care urmeaza vom introduce pe rand aceste concepte.

\begin{dfn}
Un corp finit este un corp, conform definitiei 1.12, care are multimea elementelor finita. Pe langa cele doua operatii aritmetice de baza, putem defini scaderea si impartirea pe baza operatiilor de adunare si inmultire. Astfel avem, pentru un corp $K$: $a, b\in K, a-b = a + (-b)$, unde $-b$ este inversul aditiv a lui $b$ si respectiv $a, b\in K, a/b = a \times b^{-1}$, unde $b^{-1}$ este inversul multiplicativ a lui $b$, garantat sa existe intr-un corp. 
\end{dfn}

\begin{teo}
Fie $q$ ordinul unui corp finit $\mathbb{F}$. Exista si este unic(pana la un isomorfism) un astfel de corp, daca si numai daca $q=p^{m}$, unde $p$ este caracteristica corpului $\mathbb{F}$. Daca $m=1$, atunci corpul este prim, iar daca $p\geq 2$ este corp de extensie.
\end{teo}

\begin{dfn}
Corpuri binare au caracteristica 2 si deci ordinul $2^{m}$. O metoda de a construi $\mathbb{F}/2^{m}\mathbb{F}$ este considerarea fiecare element ca un polinom cu coeficienti 0 sau 1 si cu gradul cel mult m-1. Avem astfel : 
 $\mathbb{F}/2^{m}\mathbb{F} = \set {a_{m-1}z^{m-1} + a_{m-2}z^{m-2} + ... + a_{1}z + a_{0}\mid a_{i}\in {0,1}}$
\end{dfn}



%=========================================================