\let\textcircled=\pgftextcircled

\chapter{Structuri Algebrice de bază} 
\section{Grupuri}
\label{sec:sec01}
În această secțiune vom pune bazele matematice pentru urmatoarele capitole, făcând astfel o introducere a unor concepte fundamentale din Teoria Grupurilor, a Corpurilor și în special despre Corpuri finite, care au o importanță deosebită pentru tema acestei licențe. Conceptele din acest capitol au fost preluate din \cite{algebra1} și \cite{algebra2}.

\begin{dfn}
Un \textit{grup} reprezintă o mulțime $S$ împreună cu o operație de compoziție $\cdot$ astfel încât sunt respectate următoarele reguli:
\\ - legea $\cdot$ este \textit{asociativă} $\rightarrow$ $\forall x, y, z\in S: (x\cdot y)\cdot z = x\cdot (y \cdot z)$
\\ - existența unui element \textit{neutru}, unic $\rightarrow$ $\exists e\in S: x \cdot e = x, \forall x\in S$
\\ - $\forall x\in S, \exists y\in S$, numit \textit{invers} al elementului $x$, astfel încat $x\cdot y = y\cdot x = e$
\end{dfn}

\begin{dfn}
Se numește grup \textit{abelian}, acel grup în care legea de compoziție $\cdot$ este de asemenea comutativă, adică pentru orice două elemente
$x, y \in S$ avem $x \cdot y = y \cdot x$
\end{dfn}

\begin{dfn}
Fie $G$ un grup și H o submulțime a lui $G$. $H$ este un \textit{subgrup} a lui $G$ dacă sunt îndeplinite condițiile:
\\ -închidere la operația de compoziție $\rightarrow$ $\forall x, y \in H, x \cdot y \in H$
\\ - dacă $x \in H \Rightarrow x^{-1} \in H, e \in H$
\end{dfn}

\begin{dfn}
Se numește \textit{ordin} al grupului $G$, cardinalul mulțimii G, notat $|G|$. Pentru un element $g \in G$, se numește \textit{ordin} al lui $g$, notat prin $ord_G(g)$, cel mai
mic număr natural nenul $n$, astfel încât $g^n = e$, unde $e$ este element neutru al lui $G$. Dacă nu există un astfel de număr, spunem că ordinul lui $g$ este infinit.
\end{dfn}

\begin{dfn}
Fie $G$ un grup și $g \in G$. Se numește subgrup generat de $g$ mulțimea $\set{g^n\mid n\in \mathbb{Z}}$ și se notează $\langle g \rangle$. În cazul în care G este grup finit(conține un număr finit de elemente), $G = \langle g \rangle$ dacă și numai dacă $ord_G(g) = |G|$. Atunci $G$ se numește grup \textit{ciclic} iar $g$ \textit{generator} pentru $G$.
\end{dfn}

 \begin{dfn}
 Fie $n\geq 1$. Notăm cu $\phi (n) = |(\mathbb{Z}/n\mathbb{Z})^{*}|$. Funcția $\phi$ se numește \textit{funcția lui Euler}. Avem $\phi (n) = |\set{x\mid 1\leq x\leq n, gcd(x, n)=1}|$
 \end{dfn}
 
 \begin{teo}
 Fie $n, x$ două numere întregi astfel incât $gcd(n,x)=1$. Atunci este adevarată relația:
  $x^{\phi(n)}\equiv 1 (mod n)$
 \end{teo}


%=======
\section{Inele}
\label{sec:sec02}

\begin{dfn}
Un \textit{inel} reprezintă un triplet $(R, +, \cdot)$, unde $R$ este o mulțime iar $+$ și $\cdot$ reprezintă două legi de compoziție. Următoarele proprietăți trebuiesc simultan îndeplinite: 
\\ - $(R, +)$ reprezintă un grup abelian
\\ - $x \cdot y = y \cdot x, \forall x,y\in R$ și de asemenea există element neutru la înmulțire, diferit de elementul neutru la adunare
\\ - legea $\cdot$ este \textit{distributivă} față de $+$, adică $\forall x,y,z\in R, x\cdot (y + z) = x\cdot y + x\cdot z$ și $(y + z)\cdot x = y\cdot x + z\cdot x$ 
\end{dfn}

\begin{dfn}
Fie $R, R'$ două inele cu operațiile $+, \times$ respectiv $\oplus, \otimes$. Un \textit{homomorfism} de inele este o funcție $\Psi : R\rightarrow R'$ care este definită pentru orice două elemente $x, y\in R$ astfel:
\\ - $\Psi (x+y) = \Psi (x) \oplus \Psi (y)$
\\ - $\Psi (x\times y) = \Psi (x) \otimes \Psi (y)$
\\ - $\Psi (1_R) = 1_{R'}$
\end{dfn}

\begin{dfn}
Fie $R$ un inel. $I$ Se numește \textit{ideal} (la stânga sau la dreapta) a lui $R$ dacă $I\subseteq R, I \neq \emptyset$ și respectă:
\\ - $I$ este un subgrup pentru $(R, +)$
\\ - $\forall x\in R, \forall y\in I \Rightarrow x\cdot y, y\cdot x \in I$
\\ $I$ se numește ideal \textit{bilateral} dacă este ideal la stânga și la dreapta.
\\ Idealul $I\subsetneq R$ este \textit{prim} dacă $\forall x, y\in R$ cu $x\cdot y\in I$ avem $x\in I \lor y \in I$
\\Idealul $J\subsetneq R$ este \textit{maximal} dacă pentru oricare alt ideal $J$, avem $J=I \lor J=R$
\end{dfn}

\begin{obs}
Fie $\Psi$ un homomorfism de la $\mathbb{Z}$ la un inel $R$ definit astfel:
\\ $\Psi (n) = \begin{cases} 
     1+\cdots+1 & $de n ori daca $n\geq 0 \\
   -(1+\cdots+1) & $de -n ori altfel $
   \end{cases}$
\end{obs}
 \textit{Nucleul} lui $\Psi$ este un ideal a lui $\mathbb{Z}$ și dacă toți multiplii de 1 sunt diferiți, atunci $ker\Psi = {0}$. Altfel, dacă $R$ este finit, de exemplu, câțiva multiplii vor fi $0$. Altfel spus, nucleul lui $\Psi$ este generat de un numar natural $m$
 
 \begin{dfn}
 Fie $R$ un inel și $\Psi$ definit ca mai sus. Nucleul lui $\Psi$ are forma $m\mathbb{Z}, m\in\mathbb{N}$. Elementul poartă denumirea de \textit{caracteristică} a inelului $R$ și se notează $char(R)$
 \end{dfn}
 
 \begin{dfn}
 Fie $R$ un inel. Un element $x\in R$ este inversabil daca $\exists! y\in R, x\cdot y = y\cdot x = e$. $y$ se numeste unitate. Multimea tuturor unitatilor se noteaza cu $R^{*}$
 \end{dfn}
 
 \section{Corpuri}
\label{sec:sec03}
\begin{dfn}
Un \textit{corp} $K$ este un inel comutativ cu toate elementele diferite de $0$ inversabile.
\end{dfn}

\begin{ex}
Mulțimea numerelor raționale $\mathbb{Q}$ împreună cu operațiile obișnuite de adunare si înmulțire este un corp. Pentru orice număr prim $p$, $Z_p$, împreună cu operațiile modulo p, este corp.  
\end{ex}

\begin{prop}
Caracteristica unui corp este $0$ sau $p$, un număr prim.
\end{prop}

\begin{prop}
Fie $R$ un inel și I un ideal. Inelul factor $R/I$ este corp dacă și numai dacă $I$ este maximal.
\end{prop}

\begin{dfn}
Fie $K, K'$ două corpuri. Un homomorfism de corpuri este un homomorfism de inele între $K$ si $K'$. Remarcăm faptul că o astfel de funcție este tot timpul injectivă, deoarece nucleul acesteia este $\set {0}$ 
\end{dfn}

\begin{dfn}
Fie $L$ un corp. Un \textit{subcorp} a lui $L$ este o submulțime $K$ a lui $L$ care iși păstrează proprietatea de corp, operațiile aditive respectiv multiplicative fiind moștenite de la $L$. În această situație, corpul mare, $L$, se numește o \textit{extensie} a corpului $K$.
\end{dfn}
\subsection{Corpuri Finite}
\label{subsec:subsec01}

În sistemele criptografice bazate pe curbe eliptice, este important să implementăm eficient operații pe corpuri finite. Trei tipuri de corpuri finite reprezintă canditați potriviți pentru implementarea acestor operații, respectiv corpurile prime, corpurile binare si corpurile de extensie optimale. În randurile care urmează vom introduce pe rând aceste concepte.

\begin{dfn}
Un corp finit este un corp, conform definiției 1.12, care are mulțimea elementelor finită. Pe lângă cele două operații aritmetice de bază, putem defini scăderea și împărțirea pe baza operațiilor de adunare si înmulțire. Astfel avem, pentru un corp $K$: $a, b\in K, a-b = a + (-b)$, unde $-b$ este inversul aditiv a lui $b$ și respectiv $a, b\in K, a/b = a \times b^{-1}$, unde $b^{-1}$ este inversul multiplicativ a lui $b$, garantat să existe într-un corp. 
\end{dfn}

\begin{teo}
Fie $q$ ordinul unui corp finit $\mathbb{F}$. Există și este unic(până la un isomorfism) un astfel de corp, dacă și numai dacă $q=p^{m}$, unde $p$ este caracteristica corpului $\mathbb{F}$. Dacă $m=1$, atunci corpul este prim, iar daca $p\geq 2$ este corp de extensie.
\end{teo}

\begin{dfn}
Fie \textit{p} un număr prim. Numerele naturale în mulțimea $\set {0, 1, 2, ... , p-1 }$ împreună cu operațiile de înmulțire și adunare modulo \textit{p} formează un corp finit cu ordinul p. Vom nota acest corp cu $F_p$. Pentru orice numar $a\in\mathbb{Z}$, \textit{a} mod \textit{p} reprezintă restul împarțirii lui \textit{a} la \textit{p}, număr unic în intervalul $[0, p-1]$. Această operație mai poartă denumirea de reducție modulară.
\end{dfn}

\begin{dfn}
Corpurile binare au caracteristica 2 și deci ordinul $2^{m}$. O metodă de a construi $F_{2^m}$ este considerarea fiecărui element ca un polinom cu coeficienți 0 sau 1 și cu gradul cel mult $m-1$. Avem astfel : 
 $F_{2^m} = \set {a_{m-1}z^{m-1} + a_{m-2}z^{m-2} + ... + a_{1}z + a_{0}\mid a_{i}\in {0,1}}$
\end{dfn}



%=========================================================